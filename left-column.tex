\vspace{-1.5in}
\includegraphics[width=\textwidth]{doenet-logo.pdf}

\begin{sectionblock}{The Problem}
  Despite an abundance of openly licensed educational material, \textbf{there
  isn't enough data} on how students use and those materials.  The
  absence of data inhibits our ability to develop educational
  experiences that will maximize learning.
\end{sectionblock}

\begin{sectionblock}{Goals}
  \begin{itemize}
  \item Understand how varying online experiences affect learning.
  \item Make it easier to run learning experiments; faculty should not need to set up a server.
  \end{itemize}
\end{sectionblock}
%   Generate knowledge about 
%   generate new knowledge about the effectiveness of different activities for learning
%   concepts.
  
%   the benefits of virtual manipulatives as well as the
% influence of the support given to students as they navigate the activities. We will create tools to enable
% faculty to develop their own educational experiments and generate new knowledge.

%  Its distributed architecture will facilitate participation
% without requiring one to set up a server or even have consistent internet access.

% This project will  In particular, we will investigate  Through the analysis
% of the data generated by the project, we will discover ways in which students interact with online
% materials, and look for correlations between those interactions and learning gains. Knowledge gained
% from this analysis will aid in the development of more effective online learning activities and will help us
% build tools that instructors and students can use to locate the most effective content. As the data will be
% stored in an open data warehouse on Doenet, any educational researcher can probe the data to generate
% additional knowledge on how students learn through online activities.

% Broader Impacts:
% The Distributed Open Education Network (Doenet) is designed to 



